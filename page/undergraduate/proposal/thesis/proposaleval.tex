\cleardoublepage

\ifthenelse{\equal{\MajorFormat}{cs}}
{
    \chapternonum{毕业论文(设计)文献综述和开题报告考核}
    \bfseries

    {
        \zihao{4}
        \noindent 导师对开题报告、外文翻译和文献综述的评语及成绩评定:

        该生的毕业论文选题有一定新意。
        文献综述结合了近年来几个相关方向的多篇论文,相对完整地阐述了趋势,
        并包含了该生自己对领域的理解。
        研究内容切合时代背景,具有一定研究价值。
        研究计划切实、可行,时间安排合理,工作量适中。
        开题报告撰写规范,同意开题。
    }


    \thesisproposaleval[9][14][5]
    \signature{导师签名}

    {
        \zihao{4}
        \noindent 学院盲审专家对开题报告、外文翻译和文献综述的评语及成绩评定:
        
        文献综述部分详细介绍了当前具有代表性的主流边缘端 AI 芯片,
        和神经网络结构,并提出了目前实际应用中的主要问题。
        开题报告以主流边缘端 AI 芯片作为研究对象,研究一体化的测试平台,
        满足多样化的测试需求,挖掘硬件精度对模型准确度、模型推理速度的影响,
        理解在不同神经网络部署在不同硬件上对能耗的影响因素。预期目标良好。
    }


    \mbox{} \vfill
    \thesisproposaleval[9][13][4]
    \signature{开题报告审核负责人(签名/签章)}
}
{
    \chapternonum{毕业论文(设计)文献综述和开题报告考核}
    \bfseries

    {
        \zihao{4}
        \noindent 对文献综述、外文翻译和开题报告评语及成绩评定:
    }


    \mbox{} \vfill
    \thesisproposaleval
    \signature{开题报告答辩小组负责人(签名)}
}
