\section{项目的主要内容和技术路线}

\subsection{主要研究内容}

本研究计划从四个方面研究不同神经网络在不同硬件上的执行效率,从而研究不同AI芯片的特性:
1.神经网络基本组成Operation用不同的配置在不同硬件上的执行效率;
2.神经网络基本组成Block(如MobileNet v2中提出的Inverted Residual)用不同的配置在不同硬件上的执行效率;
3.不同的SOTA模型在不同硬件上使用不同的量化精度对推理准确度、推理速度、能耗的影响;
4.其他各种Operation在硬件上的表现和影响。

本研究计划通过分析实际执行效率、能耗和理论计算复杂度之间的差别,理解边缘端硬件在并行性、资源调度方面的特性,
从而对今后边缘端友好的神经网络设计有所启发。

\subsection{技术路线}

\subsubsection{Benchmark选择}

对于针对Operation的benchmark,首先鉴于卷积操作在神经网络中占有相当大的计算量,所以本项目会重点测量卷积
操作在不同配置下的延时。另外近年来depthwise convolution在各种为效率而设计的移动端神经网络中都逐渐扮演了
更加重要的角色,所以本研究也会着重测量与分析DWConv(depthwise convolution)。除了Operation的选择之外,
比较重要的是Operation设置的选择:

Conv的配置空间:本研究计划对三种配置进行测量:
1.固定输入通道数,改变输出通道数;2.固定输出通道数,改变输入通道数;3.输入通道数等于输出通道数。
stride选取{1,2},kernel size选取{1,3,5}。对于不同的输入feature map size,本项目会选取合适频道数区间benchmark,
在平衡benchmark时间的同时尽量兼顾覆盖到当下流行网络的Operation配置。

DWConv的配置空间:stride选取{1,2},kernel size选取{3,5,7}。对于不同的feature map size,会选取不同的频道数区间。

对于针对Block的benchmark,TODO

\subsubsection{硬件测量方法}

本研究使用了若干边缘端AI芯片作为研究对象,如:Snapdragon 845/855 CPU、Adreno 630/640 GPU、高通Hexagon DSP、
Rockchip RK1808计算棒、Rockchip RK3399Pro NPU、Intel Movidius计算棒、Google Edge TPU。项目内用作推理的
TensorFlow版本为R2.1,编译命令为(--config=android\_arm64)。

Snapdragon 845/855 CPU:使用TensorFlow Lite进行benchmark。并且在测试时把CPU的频率固定在了最高频保证延时和
CPU频率无关。

Adreno 630/640 GPU:使用TensorFlow Lite提供的GPU代理进行benchmark。TensorFlow Lite的GPU代理使用了OpenCL实现。
本项目修改了部分TensorFlow Lite的GPU代理代码,使得GPU上内存拷贝,内存layout转换的时间可以和用于计算的OpenCL Kernel
运行时间分开,从而达到分开记时的效果。

Rockchip RK1808计算棒、Rockchip RK3399Pro NPU:Rockchip官方提供了唯一的闭源的神经网络编译器和神经网络推理器RKNN。
RKNN提供了对每层神经网络延时Profiling的功能。本项目在这两个硬件平台上利用RKNN进行benchmark。

Google Edge TPU:Edge TPU要求使用8-bit量化过并经过额外闭源TPU编译器的tflite模型。在经过TPU编译器优化之后,
MobileNet v2可以达到惊人的380帧。鉴于TPU在TensorFlow Lite上的代理是闭源的,本项目直接对整个代理的耗时benchmark。

\subsection{可行性分析}

针对多种不同的硬件,本项目都预先做了一些前置性、探索性的benchmark和预先的分析,判断理论与实际之间的差距是否可以
并值得分析。在这里我们把它们分成几大类:向量化的架构(Hexagon DSP)、通用处理器(Snapdragon CPU)、
SIMT(Same Instruction Multiple Threads)架构(Adreno GPU)、NPU(RK1808计算棒)。

\subsubsection{向量化硬件架构的测试与特性分析}

在可行性分析阶段,本研究预先测量了Hexagon DSP。首先我们测量了Conv Operation在各个不同的配置下的延时。
我们发现Hexagon DSP在任何的输入feature map size的条件下,固定输入通道数变化输出通道数、固定输出通道
数变化输入通道数,Conv $1\times 1$的延时都会随着通道数的变化呈现出阶梯化上升的趋势,并且这个阶梯的长度固定
为32通道。在翻阅了DSP相关官方文档开源代码之后,我们发现Hexagon DSP上的神经网络推理软件平台
SNPE(Snapdragon Neuron Processing Engine)在利用DSP汇编实现Conv $1\times 1$的时候,都会把输入的每一维:
长、宽、通道数扩展为32的倍数,这就导致了上文所阐述的阶梯状上升的延时。

\subsubsection{通用处理器的测试与特性分析}

我们预先测量了Conv $1\times 1$在高通骁龙855 CPU上的表现,经过漫长的测试,我们发现:在固定输入通道数,
改变输出通道数时,延时会表现出随着通道数阶梯状上升的趋势,并且阶梯长度固定是8;在固定输出通道数,改变输入通道数时,
延时会随着work load体现出近似线性增加。经过阅读代码之后,我们发现TensorFlow在Arm 64的环境下利用了NEON加速实现
了自己的GEMM(通用矩阵计算)库RUY,Conv在im2col之后使用了RUY做计算。而RUY在实现矩阵乘法的时候,为了更好地利
用NEON的SIMD特性,在输出通道数的维度上扩展到8的倍数,而在输入通道数上没有额外的扩展。因而体现出了在输出通道数
上有延时台阶式上升的现象。

\subsubsection{SIMT架构硬件的测试与特性分析}

我们在Adreno 640上测试了Conv $1\times 1$作为SIMT架构的代表。我们修改了TensorFlow Lite的代码使得我们可以
指定每一层的OpenCL Kernel的work group size或者是TensorFlow Lite GPU代理的work group size调优方式。我们
固定调优方式为exhaustive。经过测试我们发现当固定输入通道数,修改输出通道数的时候延时随着通道数有阶梯式的上
升,而且这个阶梯长度在输入长宽为7时长达几百通道数。但是固定输出通道数修改输入通道数时,延时却随着通道数近似
线性地增加(每4个通道一采样)。阅读过卷积操作的OpenCL Kernel实现之后,我们发现卷积的实现是在输入通道上
迭代,而在输出通道上划分work group。因为移动端GPU受限于compute unit数量、每work item的private memory大小,
每线程占有寄存器数量,所以在GPU上一个批次执行的work group数量是有限的。所以当修改输出通道数时,延时会体现出
批次化上升的现象。

\subsubsection{NPU架构硬件的测试与特性分析}

我们预先测量了RK1808计算棒在Conv $1\times 1$上的表现。我们发现计算棒不论在固定输入通道数还是在固定输出通道数
的情况下延时都会表现出比较线性的随通道数增长,但是存在许多配置,延时会偏离化得极高或者极低。由于RK1808上的
VIP8000 NPU不开源,架构信息偏少,编译器不开源,延时曲线背后的原因暂时没有合理的解释。