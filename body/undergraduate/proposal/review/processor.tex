\section{边缘端神经网络推理设备}
本节简要介绍了一些主流的边缘端神经网络推理设备,和相关的基准测试的研究。

\paragraph{移动端CPU/GPU}
移动端CPU和GPU通常集成在SoC上出现。
以安卓端的生态为例,常见的SoC有高通骁龙、华为麒麟系列等。
麒麟SoC搭载了公版Arm的Cortex CPU和Mali GPU;
而骁龙上搭载的Adreno GPU来自高通收购的ATI移动GPU部门Imageon。
Arm CPU的一个特点是具有称为NEON的SIMD加速单元,可以用于图像处理和AI应用加速。
而Mali和Adreno GPU都属于SIMT架构,每个GPU线程可以并行而独立地处理数据和控制流。

\paragraph{Hexagon DSP}
Hexagon DSP是高通处理器的关键组件。
在信号处理和多媒体应用程序上,Hexagon DSP提供了更有竞争力的性能和电源效率。
它具有硬件多线程,采用了可变长度指令和超长指令字(VLIW)的处理器体系结构。
现代Hexagon DSP使用优先级调度策略以更好地隐藏延时。

\paragraph{Edge TPU}
Google Edge TPU是采用脉动阵列架构的矩阵乘法器。
一个脉动阵列往往由多个同构的处理单元组成,数据在处理单元之间流动。
每个处理单元的设计相对简单,系统通过实现大量处理单元并行来提高运算的效率。
TPU采用这样的设计,一方面是可以大幅度提高并行度,另一方面是可以平衡运算和IO。

\subsection{针对边缘端处理器的基准测试}
许多不同的工作都已经以不同目的对各种边缘端处理器的AI处理能力进行过基准测试。
BenchIP\cite{tao2017benchip}和DeepBench\cite{deepbench}在一个基准测试套件下,测试推理AI任务的硬件设备的性能。
前者的工作将测试分为宏观测试(Macro Benchmark)和微观测试(Micro Benchmark)。
其中,宏观测试测试了常见的神经网络模型,而微观测试测试了由常见的基本算子构成的单层网络。
这样做旨在在覆盖常见的使用场景的前提下,也帮助设计者利用特殊的配置数据探索新的研究方向。
宏观测试比较了网络模型之间的相似性和相关度;
微观测试在不同的硬件上以不同的配置测试了单层网络,文章也分析了不同算子的数据重用距离以分析算子的数据局部性。