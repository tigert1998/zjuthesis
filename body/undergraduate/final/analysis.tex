\newcounter{finding}[section]
\newenvironment{finding}[1][]{
    \refstepcounter{finding}\par
    \medskip\textbf{发现\thefinding#1} 
    \rmfamily}{\medskip}

\section{硬件行为分析}
\label{analysis}
本节分析了七种处理器上的基准测试结果,并展示了神经网络与硬件行为的发现。
本文会对不寻常的结果,基于硬件特性、框架实现和神经网络结构给出详尽的解释。
评估维度包括通道数(节\ref{analysis:channels})、
模块类型、卷积内核尺寸、激活函数(节\ref{analysis:op block})和
量化算法(节\ref{analysis:model quantization})。

\subsection{通道数}
\label{analysis:channels}

\inputbody{final/analysis_misc/conv_pics}

通道数是对高效神经网络调优的重要参数。
基本的一个直觉是更大的通道数意味着更多计算数,因此有着更长的推理时间。
本节将展示通道数变化时真实的卷积延时响应。

图\ref{fig:conv_cin_fixed}展示了在各个硬件上变化$C_{out}$导致的卷积延时变化(固定$C_{in}$和其他超参数)。

\begin{finding}
    除了KPU之外,卷积延时随着输出通道数以台阶的规律增长。
\end{finding}

\subsection{算子和模块}
\label{analysis:op block}

\subsection{模型与量化算法}
\label{analysis:model quantization}