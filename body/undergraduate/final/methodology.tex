\section{基准测试套件和测量方法}
\label{methodology}
本文的目的是定量地分析设计空间中各个维度上的神经网络硬件的行为,
来为在边缘设备上构建高效网络提供全面的见解和改进方针。
合理的基准测试选择(节\ref{suite})和严格的测量方法(节\ref{measurement})是实现此目标的前提。

\subsection{基准测试套件}
\label{suite}

\paragraph{通道数}
如节\ref{nn design and deployment}所述,通道数对于边缘设备上的模型可用性至关重要。
除了启发式的规则之外,基于剪枝和搜索的方法都广泛使用OPs作为通道数的一个硬约束,
并假设较少的通道数(即较少的OPs)意味着更快的推理速度。
但是真实设备上的推理延时是否符合这样的假设呢?
为了回答这个问题,表todo所列的基准测试套件内包含通道数的类别。

此类别主要关注卷积(Conv)和逐通道卷积(DWConv)算子,因为它们占据了CNN推理的大部分延时。
它包含三种配置:
\begin{enumerate*}[label=(\alph*)]
    \item 固定输入通道数$C_{in}$,并变化输出通道数$C_{out}$;
    \item 固定输出通道数$C_{out}$,并变化输入通道数$C_{in}$;
    \item 在逐通道卷积上变化通道数($C_{in}=C_{out}$)。
\end{enumerate*}
通道数的范围是从主流的高效CNN模型中总结出来的。
该类别也包括了特征图尺寸(从$224^2$到$7^2$)和内核尺寸(1,3,5,7)的维度。

\paragraph{算子和模块}
算子和模块是神经网络模型设计的核心。
我们的基准测试套件包含了丰富的网络结构池。

在算子方面,套件包括常用的逐元素算子,卷积以及一系列激活函数。
ReLU、ReLU6和Sigmoid是传统的激活函数。
近年来Swish\cite{ramachandran2017swish}也被用于代替ReLU以提高准确率。
HardSwish用近似的方法来降低Swish的计算复杂性。
在模块方面,套件包括了节\ref{nn design and deployment}讨论的来自SOTA模型的有代表性的模块(如图todo所示)。
套件还包含不同的特征图尺寸和内核尺寸以进行综合分析。

影响神经网络硬件行为的一个重要因素是
算子的数据重用率(Data Reuse Rate),即计算密度(Operational Intensity)。
表todo显示了典型算子具有的OPs和内存访问开销(Memory Access Cost)的表达式。
显然,卷积是最计算密集的算子,因而其性能往往受硬件计算带宽的限制。
对于分组卷积,因为其特征图被按组($G$)划分计算以减少OPs,所以其数据重用率较低。
DWConv可以看作$G = C_{in}$的分组卷积。
它的数据重用率常是Conv的几十分之一。
逐元素算子和激活函数对张量的每个元素都执行例如加或乘的操作。
他们是最内存密集型的算子,其性能对AI加速器的内存带宽非常敏感。

影响硬件行为的另一个因素是模块结构。
例如,逐元素算子在网络中的位置决定了它们是否可以与Conv融合以减少内存访问。
我们将在节todo内详细解释这些因素是如何影响神经网络行为的。

\paragraph{模型与量化}
套件内还包含了14个CNN模型,包括了最具代表性的手工设计的模型和NAS搜索的模型。
\begin{itemize}
    \item 手工设计的模型:适用于移动端设备的MobileNetV1/V2、ShuffleNetV1/V2,
    以及大模型如ResNetV1/V2、InceptionV1\cite{szegedy2015going}。
    \item 搜索得到的模型:MobileNetV3-Large-1.0\cite{howard2019searching}、
    Proxyless-Mobile/Mobile-14\cite{cai2018proxylessnas}、
    MnasNet-A1\cite{tan2019mnasnet}、
    NasNet-A-Mobile\cite{zoph2018learning}、
    EfficientNet-B0/B1\cite{tan2019efficientnet}
\end{itemize}
这些模型具有不同的计算、访存复杂度。
它们的操作数从140 MOPs(ShuffleNetV1)到6.55 GOPs(ResNetV2)不等。
模型大小从9 MB(ShuffleNetV2)到99 MB(ResNetV2)不等。
基准测试包括了各硬件支持的所有精度,例如AI加速器上的FP16和INT8(表todo),
来评估量化算法对各模型加速比和准确度的影响。

\subsection{测量方法}
\label{measurement}