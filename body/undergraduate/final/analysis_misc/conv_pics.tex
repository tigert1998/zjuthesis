\begin{figure}[htbp]
    \centering
    \includegraphics[width=0.96\textwidth]{final/cpuconv}
    \caption{TFLite ruy中的矩阵乘法实现。橙色标注了计算的基本模块。数据被扩展到了8的倍数。}
    \label{fig:cpuconv}
\end{figure}

\begin{figure}[htbp]
    \centering
    \includegraphics[width=0.8\textwidth]{final/gpuconv}
    \caption{TFLite OpenCL后端的卷积实现。虚线划分了工作组。橙色显示了输出张量中的一个元素涉及的计算区域。}
    \label{fig:gpuconv}
\end{figure}

\begin{figure}[htbp]
	\centering
	\includegraphics[width=0.95\textwidth]{final/analysis/conv_cin_fixed}
	\caption{
        除了KPU之外,卷积延时随着$C_{out}$变化显示出了台阶的规律。
        本图所用的配置是:
        $H \times W=28 \times 28$,$C_{in}=320$,$K=3$,$S=1$。
        X轴是$C_{out}$(为了更好地显示规律,不同的子图使用了不同的间隔),而Y轴是以毫秒记的延时。
	}
	\label{fig:conv_cin_fixed}
	\phantomlabel{a}{fig:conv_cin_fixed:cpu}
	\phantomlabel{b}{fig:conv_cin_fixed:gpu}
	\phantomlabel{c}{fig:conv_cin_fixed:dsp}
	\phantomlabel{d}{fig:conv_cin_fixed:vpu}
	\phantomlabel{e}{fig:conv_cin_fixed:tpu}
	\phantomlabel{f}{fig:conv_cin_fixed:npu}
	\phantomlabel{g}{fig:conv_cin_fixed:kpu}
\end{figure}