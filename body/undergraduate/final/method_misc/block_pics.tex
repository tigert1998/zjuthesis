\begin{figure}[htbp]
    \centering
    \begin{subfigure}[b]{0.1696\linewidth}
    	\centering\includegraphics[width=\textwidth]{final/blocks/resnetv1}
    	\caption{\label{fig:blocks:resnet}ResNetV1}
	\end{subfigure}
	\hspace{2.7em}
    \begin{subfigure}[b]{0.1696\linewidth}
    	\centering\includegraphics[width=\textwidth]{final/blocks/densenet}
    	\caption{\label{fig:blocks:densenet}DenseNet}
	\end{subfigure}
	\hspace{2.7em}
	\begin{subfigure}[b]{0.1808\linewidth}
		\centering\includegraphics[width=\textwidth]{final/blocks/mobilenetv1}
		\caption{\label{fig:blocks:mobilenetv1}MobileNetV1}
	\end{subfigure}
	\hspace{2.7em}
	\begin{subfigure}[b]{0.2\linewidth}
		\centering\includegraphics[width=\textwidth]{final/blocks/mobilenetv2}
		\caption{\label{fig:blocks:mobilenetv2}\small{MobileNetV2$^+$}}
	\end{subfigure}
	\hspace{2.7em}
	\begin{subfigure}[b]{0.184\linewidth}
		\centering\includegraphics[width=\textwidth]{final/blocks/shufflenetv1}
		\caption{\label{fig:blocks:shufflenetv1}ShuffleNetV1}
	\end{subfigure}
	\hspace{2.7em}
	\begin{subfigure}[b]{0.1872\linewidth}
		\centering\includegraphics[width=\textwidth]{final/blocks/shufflenetv2}
		\caption{\label{fig:blocks:shufflenetv2}ShuffleNetV2}
	\end{subfigure}
	\hspace{2.7em}
	\begin{subfigure}[b]{0.1808\linewidth}
		\centering\includegraphics[width=\textwidth]{final/blocks/se}
		\caption{\label{fig:blocks:se}SE}
	\end{subfigure}
    \caption{
        SOTA CNN模型中的有代表性的模块。
        \footnotesize{\normalfont{$^+$:对于MobileNetV2+SE模块,
        我们采用\parencite{howard2019searching}中的构建方法,
		将SE插在逐通道卷积和最后的$1\times 1$卷积之间。}}
	}
	\label{fig:blocks}
\end{figure}
