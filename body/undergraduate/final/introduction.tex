\cleardoublepage

\section{引言}
许多边缘设备,如智能手机、相机、扬声器和传感器为神经网络技术提供了理想的实际使用场景,
例如语音助手,智能安防和自动驾驶。
与将数据发送到云端服务器相比,在端侧上的推理对于保护用户隐私,以及构建始终可用且响应迅速的AI至关重要。
因此,将神经网络部署在端侧上的需求和动力是巨大的。

近年来,端侧神经网络部署的整个技术栈都取得了巨大的进展,包括硬件、推理框架和高效的神经网络结构。
其中在AI硬件上,不少传统科技公司或创业公司凭借他们的产品占据了一席之地,
例如Google的Edge TPU\cite{edgetpu},英特尔的Movidius VPU\cite{myriad},Canaan的KPU\cite{k210}
和寒武纪的MLU\cite{cambricon}。
这是整个半导体领域第一次经历这种大规模的扩张和快速迭代。
边缘端AI开发框架的发展也是迅速的,例如开源且被广泛使用的TensorFlow Lite(TFLite)\cite{tflite}
和PyTorch Caffe2\cite{caffe2},以及专用硬件供应商为其芯片产品提供的闭源推理框架等。
例如,高通开发的用于骁龙SoC的SNPE(骁龙神经处理引擎)运行时\cite{snpe}。
硬件和软件框架的多样性给神经网络算法设计带来了巨大的挑战。

\begin{figure}[htbp]
    \centering
    \includegraphics[width=.8\linewidth]{final/pipeline}
    \caption{\label{fig:pipeline}标准的端侧神经网络算法部署流程}
\end{figure}

本质上,神经网络设计的重点是提高其精确度。
如图\ref{fig:pipeline}所示,神经网络的设计空间涉及了许多维度。
为了在边缘设备上高效推理神经网络,设计空间中也要加入为效率而考虑的约束。
但是,人们对目前约束有一定的误解。
就比如最常见的约束:计算操作数
(OPs\footnote{因为不同AI芯片会采用不同的推理精度,所以本文使用OPs而不是FLOPs。
这里OPs的定义和\cite{zhang2018shufflenet}一致,即乘加的数量。})。
许多工作只是旨在通过减少OPs来降低延时。
不幸的是,较少的OPs并不直接意味着延时的减少,相反可能会损害模型的准确性。
例如,我们的测量表明,在Edge TPU上,MobileNetV3(209MOPs)比MobileNetEdgeTPU(990 MOPs)慢,并且模型精确度更低。
另一个误解是,一个高效的模型可以在每种设备都做到快速推理。
同样,事实并非如此。
根据我们的数据,MobileNetV3在ARM Cortex A76上的运行速度比MobileNetV2快25\%,
但前者在Movidius VPU比后者慢71\%。

某些神经网络设计的工作\cite{dai2019chamnet, tan2019mnasnet, yang2018netadapt}注意到了这些误解,
并在设计过程中测量了目标硬件上的实际延迟作为优化目标。

\subsection{背景}

\subsubsection{节标题}
